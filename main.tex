% This must be in the first 5 lines to tell arXiv to use pdfLaTeX, which is strongly recommended.
\pdfoutput=1
% In particular, the hyperref package requires pdfLaTeX in order to break URLs across lines.

\documentclass[11pt]{article}

% Remove the "review" option to generate the final version.
\usepackage[review]{acl}

% Standard package includes
\usepackage{times}
\usepackage{latexsym}

% For proper rendering and hyphenation of words containing Latin characters (including in bib files)
\usepackage[T1]{fontenc}
% For Vietnamese characters
% \usepackage[T5]{fontenc}
% See https://www.latex-project.org/help/documentation/encguide.pdf for other character sets

% This assumes your files are encoded as UTF8
\usepackage[utf8]{inputenc}

% This is not strictly necessary, and may be commented out,
% but it will improve the layout of the manuscript,
% and will typically save some space.
\usepackage{microtype}

% Custom
\usepackage{adjustbox}
\usepackage{graphicx}
\graphicspath{ {./images/} }
\usepackage{multirow}
\usepackage{booktabs}
\usepackage{graphics}
\usepackage{parskip}
\usepackage{amssymb}
\usepackage{amsmath}
\usepackage{float}
\usepackage{cleveref}
\usepackage{enumitem}
\usepackage{booktabs}
\usepackage{xspace}
\usepackage{makecell}
\usepackage{blindtext}
\usepackage[noend,linesnumbered,ruled,vlined]{algorithm2e}


% Commands
\newcommand\todo[1]{\textcolor{red}{TODO: #1.}}
%\newcommand{\our}{Name\xspace} TODO: have a name later

% If the title and author information does not fit in the area allocated, uncomment the following
%
%\setlength\titlebox{<dim>}
%
% and set <dim> to something 5cm or larger.

\title{Animal Crossing: Learning Faithfulness and Factuality Evaluation \\ via Large Language Models}

% Author information can be set in various styles:
% For several authors from the same institution:
% \author{Author 1 \and ... \and Author n \\
%         Address line \\ ... \\ Address line}
% if the names do not fit well on one line use
%         Author 1 \\ {\bf Author 2} \\ ... \\ {\bf Author n} \\
% For authors from different institutions:
% \author{Author 1 \\ Address line \\  ... \\ Address line
%         \And  ... \And
%         Author n \\ Address line \\ ... \\ Address line}
% To start a seperate ``row'' of authors use \AND, as in
% \author{Author 1 \\ Address line \\  ... \\ Address line
%         \AND
%         Author 2 \\ Address line \\ ... \\ Address line \And
%         Author 3 \\ Address line \\ ... \\ Address line}

\author{First Author \\
  Affiliation / Address line 1 \\
  Affiliation / Address line 2 \\
  Affiliation / Address line 3 \\
  \texttt{email@domain} \\\And
  Second Author \\
  Affiliation / Address line 1 \\
  Affiliation / Address line 2 \\
  Affiliation / Address line 3 \\
  \texttt{email@domain} \\}

\begin{document}
\maketitle
\begin{abstract}
Hallucination detection aims to evaluate if the generated text is consistent with the inputs (i.e.,~faithfulness) or consistent with common sense (i.e.,~factuality).
Previous model-based hallucination detection methods usually focus on a single task (e.g.,~summarization) and a single purpose (i.e.,~faithfulness or factuality) because supervisions used for training need lots of manual annotations or heuristics rules for dataset construction. To reduce human efforts, previous work employed a closed large language model (LLM) to annotate summarization data and then trained a model-based metric. However, they still have a limited scope and 
%a closed LLM is hard to follow. 
it is unclear how open-source LLMs perform on this task.
Hence, in this paper, to have a thorough understanding of using supervision from LLMs, we first conduct a comprehensive LLM evaluation on multiple tasks and purposes. Then, with supervision from LLMs, we learn a model-based metric that can be used for both faithfulness and factuality on multiple NLP tasks. Extensive experiments show that \todo{experiments}
\end{abstract}
\section{Introduction}
Text generation models have demonstrated promising real-world applications. One core issue of text generation is models often generate text that is unfaithful, untrue or even completely "hallucinated"~\cite{Rohrbach2018ObjectHI, maynez-etal-2020-faithfulness, Honovich2022TRUERF}. In terms of consistency types, we can divide evaluating hallucination into two purposes,~\emph{faithfulness} and \emph{factuality}. The two purposes are usually used for different tasks. As shown in~\todo{Figure 1}, faithfulness evaluates the consistency between the generated text and input documents for tasks, e.g.,~summarization~\cite{Nallapati2016AbstractiveTS}, OpenQA~\cite{Hermann2015TeachingMT}; factuality evaluates the consistency between the generated text and facts (or common sense) for tasks, e.g.,~common-sense QA~\cite{Zellers2018SWAGAL}. 
\section{Task Definition and Benchmark}
\subsection{Task Definition}
In this paper, we aim to evaluate faithfulness and factuality using one model-based metric trained on annotations from large language models. Given generated text from models, we define \textit{faithfulness} and \textit{factuality} as follows:

\textbf{Faitfhulness}: given a grounding text, \textit{faithfulness} evaluates if the generated text is consistent with the corresponding grounding text.

\textbf{Factuality}: without grounding text, \textit{factuality} evaluates if the generated text is consistent with common sense and facts in world knowledge.

As shown in \todo{Fig}, when both grounding text and generated text are given, our metric will evaluate \textit{faithfulness}; when only generated text is given, our metric will evaluate \textit{factuality}. For each data instance, the metric outputs a score which will range from 0 to 1 and 0 represents the generated text containing hallucination; 1 represents the generated text is faithful to inputs or consistent to common sense.


\subsection{Benchmark}
\label{sec:benchmark}
In this section, we introduce datasets used for \textit{faithfulness} and \textit{factuality} evaluation.
\subsubsection{Faithfulness}
\label{sec:bench_faithfulness}
We include 13 datasets that contain human annotations w.r.t faithfulness in diverse NLP tasks including summarization, paraphrasing, fact-checking, dialogue and open QA. The numbers for these datasets are shown in~\Cref{tab:faithfulness_data}. For HaluEval datasets (i.e.,~HaluEval-sum, HaluEval-dial and HaluEval-QA), we directly use the annotations from datasets. But for other datasets, to obtain the faithfulness annotations, we follow previous work~\cite{Honovich2022TRUERF} to preprocess datasets. Then, in all processed datasets, each data instance has one \textit{Grounding Text}, one \textit{Generated Text} and a corresponding label. If the \textit{Generated Text} is faithful to the \textit{Grounding Text}, then the label will be 1, otherwise, the label is 0. Detailed dataset introduction can be found in~\todo{Appendix}.

\begin{table}[t]
\small
\centering
\scalebox{0.95}{
\setlength{\tabcolsep}{1mm}{
\begin{tabular}{lrr}
\toprule
\multicolumn{1}{c}{\textbf{Dataset}} & \multicolumn{1}{c}{\textbf{\# Examples}} & \multicolumn{1}{c}{\textbf{\# Positives}} \\ \midrule
\multicolumn{3}{c}{\textbf{Summarization}} \\ \midrule                                   
Frank~\cite{pagnoni-etal-2021-understanding}                                & 671                                      & 33.2\%                                     \\
Summeval~\cite{Fabbri2020SummEvalRS}                             & 1,600                                    & 81.6\%                                     \\
MNBM~\cite{maynez-etal-2020-faithfulness}                                 & 2,500                                    & 10.2\%                                     \\
QAGS~\cite{wang-etal-2020-asking}                                 & 239                                      & 48.5\%                                     \\
HaluEval-sum~\cite{Li2023HaluEvalAL}                         & 20,000                                   & 50.0\%                                     \\ \midrule
\multicolumn{3}{c}{\textbf{Dialogue}}        \\ \midrule                                 
BEGIN~\cite{Dziri2021EvaluatingAI}                                & 836                                      & 33.7\%                                     \\
Q$^2$~\cite{honovich-etal-2021-q2}                                   & 1,088                                    & 57.7\%                                     \\
DialFact~\cite{Gupta2021DialFactAB}                             & 8,689                                    & 38.5\%                                     \\
HaluEval-dial~\cite{Li2023HaluEvalAL}                        & 20,000                                   & 50.0\%                                     \\ \midrule
\multicolumn{3}{c}{\textbf{Fact Check}}                                                                                      \\ \midrule   
Fever~\cite{thorne-etal-2018-fact}                                & 18,209                                   & 35.1\%                                     \\
VitaminC~\cite{schuster-etal-2021-get}                             & 63,054                                   & 49.9\%                                     \\ \midrule
\multicolumn{3}{c}{\textbf{Paraphrasing}}                                                                                    \\ \midrule   
PAWS~\cite{zhang-etal-2019-paws}                                 & 8,000                                    & 44.2\%                                     \\ \midrule
\multicolumn{3}{c}{\textbf{QA}}                                                                                              \\ \midrule   
HaluEval-QA~\cite{Li2023HaluEvalAL}                          & 20,000                                   & 50.0\%                                     \\ \bottomrule
\end{tabular}
}}
\caption{Dataset statistics for faithfulness.}
\label{tab:faithfulness_data}
\end{table}

\subsubsection{Factuality}
\begin{table}[t]
\small
\centering
\scalebox{0.87}{
\setlength{\tabcolsep}{1mm}{
\begin{tabular}{lrr}
\toprule
\multicolumn{1}{c}{\textbf{Dataset}} & \multicolumn{1}{c}{\textbf{\# Examples}} & \multicolumn{1}{c}{\textbf{\# Positives}} \\ \midrule
\textbf{Wiki bio}~\cite{manakul2023selfcheckgpt}                    & 476                                      & 52.5\%                                     \\ \midrule
\textbf{Halu-general}~\cite{Li2023HaluEvalAL}                & 5,000                                    & 80.5\%                                     \\ \midrule
\textbf{TrueFalse}~\cite{Azaria2023TheIS}                   &                                          &                                            \\
- Animals                             & 1,008                                    & 50.0\%                                     \\
- Capitals                           & 1,458                                    & 50.0\%                                     \\
- Cities                             & 10,000                                   & 51.6\%                                     \\
- Companies                          & 2,850                                    & 45.4\%                                     \\
- Elements                           & 930                                      & 50.0\%                                     \\
- Facts                              & 2,295                                    & 44.2\%                                     \\
- Inventions                         & 876                                      & 53.0\%                                     \\ \bottomrule
\end{tabular}
}}
\caption{Dataset statistics for factuality.}
\label{tab:factuality_data}
\end{table}

We include three kinds of factuality datasets in this paper: (1)~\textbf{Wiki bio} contains 476 instances which are Wikipedia-like passages generated from GPT-3~\footnote{text-davinci-003} using the prompts. Authors manually annotate sentence-level factuality. In our experiments, only when all sentences are factual we consider a passage factual (the label is 1)~\footnote{We consider all original wiki bio passages (from Wikipedia instead of model generation) as factual text.}. (2)~\textbf{Halu-general} contains 5,000 human-annotated samples for ChatGPT responses to general user queries from Alpaca~\cite{alpaca}. (3)~\textbf{TrueFalse} contains 7 categories of short-sentence statements about facts in the world. For each statement, we have a label 0 or 1 for factuality.
\section{LLM Evaluation and Analysis}
To study abilities of different open-source LLMs on evaluating faithfulness and factuality and select the best LLM as our annotator, we first conduct comprehensive experiments of LLMs on the benchmark datasets introduced in~\Cref{sec:benchmark}.
\subsection{Models}
There are various large language models proposed in previous work, and we select the most powerful models when we write this paper for evaluation and analysis:
\textbf{Open-source models}:

(1) \textbf{Dolly v2} (7B, 12B)~\cite{DatabricksBlog2023DollyV2};
(2) \textbf{Falcon} (7B, 40B)~\cite{falcon40b};
(3) \textbf{Koala} (7B, 13B)~\cite{koala_blogpost_2023};
(4) \textbf{Alpaca} (7B, 13B)~\cite{alpaca};
(5) \textbf{Vicuna} (7B, 13B)~\cite{vicuna2023};
(6) \textbf{Llama2} (7B, 13B, 70B)~\cite{Touvron2023Llama2O}.

For Falcon and Llama2, we use the instruction-finetuned version and chat version respectively, because we find their original versions will not follow our prompt to output usable answers as annotations.

\textbf{Closed-source models}:

(1) \textbf{GPT-3.5-Turbo}~\footnote{\url{https://platform.openai.com/docs/models/gpt-3-5}}; 
(2) \textbf{GPT-4}~\footnote{\url{https://platform.openai.com/docs/models/gpt-4}}.

We introduce checkpoint details of LLMs and conversational templates used for LLMs in \todo{Appendix}.

\subsection{Faithfulness}
\label{sec:llm_faithful}
\textbf{Datasets}
To efficiently evaluate LLMs on faithfulness, we randomly sample 100 examples from each dataset introduced in~\Cref{sec:bench_faithfulness}, hence, we have in total 1,300 data instances for faithfulness evaluation. 

\textbf{Experiment Settings}
We feed each instance into a zero-shot prompt template to query LLMs to get answers:

\textit{Input Text: <\textbf{grounding text}>\\
Generated Text: <\textbf{generated text}>\\
Can this generated text be inferred from the input text? Please have a thoughtful and rigorous consideration of this question and answer in ``True'' or ``False''.}

LLMs will generate answers containing ``True'' or ``False'' and we parse answers into labels 1 or 0 respectively. We use zero-shot prompting since applying few-shot prompting did not improve performance in early experiments. This observation is consistent with TrueTeacher~\cite{Gekhman2023TrueTeacherLF} results.

We treat this task as a binary classification problem and LLMs will output binary predictions ~\todo{K: Response Parsing in Appendix}. To evaluate performance, we use \textit{ungrounded} F1 scores as the evaluation metric which means we treat ungrounded label (i.e.,~label 0 in the data) as positives for F1 calculation for evaluating the ability of LLMs to detect hallucinations.




% Please add the following required packages to your document preamble:
% \usepackage{booktabs}
% \usepackage[table,xcdraw]{xcolor}
% If you use beamer only pass "xcolor=table" option, i.e. \documentclass[xcolor=table]{beamer}
\begin{table*}[t]
\centering
\small
\scalebox{0.98}{
\setlength{\tabcolsep}{1mm}{
\begin{tabular}{l|ccccc|cccc|cc|c|c|c}
\toprule
\multicolumn{1}{c|}{} &
  \multicolumn{5}{c|}{\textbf{Summarization}} &
  \multicolumn{4}{c|}{\textbf{Dialogue}} &
  \multicolumn{2}{c|}{\textbf{Fact Check}} &
  \textbf{Para.} &
  \textbf{QA} &
   \\ \cmidrule(lr){2-14}
\multicolumn{1}{c|}{\multirow{-2}{*}{\textbf{LLMs}}} &
  Frank &
  Sum &
  MNBM &
  QAGS &
  H-sum &
  BEGIN &
  Q2 &
  DialFact &
  H-dial &
  Fever &
  VitaC &
  PAWs &
  H-QA &
  \multirow{-2}{*}{\textbf{Overall}} \\ \midrule
Dolly v2-7B  & 0.11  & 0.07 & 0.14 & 0.11 & 0.05  & 0.14  & 0.32 & 0.17     & 0.25   & 0.40  & 0.21  & 0.29 & 0.27 & 0.20 \\
Dolly v2-12B & 0.50  & 0.39 & 0.32 & 0.29 & 0.28  & 0.37  & 0.49 & 0.57     & 0.42   & 0.66  & 0.46  & 0.38 & 0.38 & 0.45 \\
Falcon-7B    & 0.17  & 0.07 & 0.26 & 0.20 & 0.23  & 0.16  & 0.20 & 0.10     & 0.27   & 0.23  & 0.20  & 0.26 & 0.16 & 0.20 \\
Falcon-40B   & 0.71  & 0.34 & 0.63 & 0.46 & 0.40  & 0.54  & 0.36 & 0.62     & 0.50   & 0.63  & 0.46  & 0.60 & 0.44 & 0.53 \\
Koala-7B     & 0.60  & 0.22 & 0.46 & 0.27 & 0.18  & 0.46  & 0.55 & 0.61     & 0.35   & 0.64  & 0.47  & 0.48 & 0.26 & 0.46 \\
Koala-13B    & 0.69  & 0.26 & 0.79 & 0.53 & 0.53  & 0.70  & 0.60 & 0.74     & 0.59   & 0.79  & 0.59  & 0.66 & 0.46 & 0.63 \\
Alpaca-7B    & 0.55  & \textbf{0.49} & 0.57 & 0.55 & 0.16  & 0.55  & 0.55 & 0.76     & 0.39   & 0.78  & 0.54  & 0.44 & 0.48 & 0.55 \\
Alpaca-13B   & 0.70  & 0.39 & 0.74 & 0.52 & 0.45  & 0.55  & 0.64 & 0.74     & 0.63   & 0.77  & 0.41  & 0.60 & 0.54 & 0.61 \\
Vicuna-7B    & 0.50  & 0.20 & 0.58 & 0.28 & 0.29  & 0.53  & 0.60 & 0.62     & 0.38   & 0.75  & 0.63  & 0.48 & 0.42 & 0.52 \\
Vicuna-13B   & 0.81  & 0.31 & 0.77 & 0.64 & 0.49  & 0.61  & 0.65 & \textbf{0.86}     & 0.68   & 0.84  & 0.64  & 0.67 & 0.75 & 0.70 \\
Llama2-7B    & 0.81  & 0.33 & 0.86 & 0.65 & 0.59  & 0.76  & 0.58 & 0.74     & 0.55   & 0.83  & 0.61  & 0.74 & 0.68 & 0.70 \\
Llama2-13B   & \textbf{0.87}  & 0.38 & \textbf{0.95} & \textbf{0.69} & \textbf{0.66}  & 0.60  & 0.60 & 0.77     & \textbf{0.70}   & 0.81  & \textbf{0.71}  & 0.76 & 0.73 & \textbf{0.73} \\
Llama2-70B   & 0.71  & 0.41 & 0.78 & 0.57 & 0.43  & \textbf{0.75}  & \textbf{0.69} & 0.83     & 0.68   & \textbf{0.86}  & 0.68  & \textbf{0.84} & \textbf{0.76} & \textbf{0.73} \\ \midrule
GPT3.5       & \underline{0.87}  & 0.47 & \underline{0.87} & \underline{0.71} & \underline{0.61}  & 0.77  & 0.71 & 0.82     & 0.68   & 0.88  & 0.65  & 0.69 & 0.76 & 0.75 \\
GPT4         & \underline{0.87}  & \underline{0.50} & 0.79 & 0.67 & 0.33  & \underline{0.78}  & \underline{0.75} & \underline{0.88}     & \underline{0.74}   & \underline{0.89}  & \underline{0.70}  & \underline{0.80} & \underline{0.83} & \underline{0.80} \\ \bottomrule
\end{tabular}
}}
\caption{The performance (F1 scores) of faithfulness evaluation using different LLMs over 13 datasets. The best score on each dataset is \textbf{bold} for open-source LLMs and \underline{underlined} for closed-source LLMs.}
\label{tab:llm_eval_faithful}
\end{table*}

\textbf{Result Analysis}


\subsection{Factuality}
\textbf{Datasets}
Similar as faithfulness evaluation, we randomly sample 100 examples from different factuality datasets. For TrueFalse dataset, we sample 100 examples from each category. We have 1,000 data instances for LLM evaluation in total.

\textbf{Experiment Settings}
We feed each instance into a zero-shot prompt template to query LLMs to get answers:

\textit{Generated Text: <\textbf{generated text}>\\
Is the generated text consistent to the common sense or facts in the world? Please have a thoughtful and rigorous consideration of this question and answer in ``True'' or ``False''.}

Similarly, LLMs generate answers containing ``True'' or ``False'' and we parse answers into labels 1 or 0 respectively. We do not see improvements with few-shot prompts for this task.


% Please add the following required packages to your document preamble:
% \usepackage{booktabs}
\begin{table*}[]
\centering
\small
\scalebox{0.98}{
\setlength{\tabcolsep}{1mm}{
\begin{tabular}{l|ccccccccc|c}
\toprule
\multicolumn{1}{c|}{\textbf{LLMs}} &
  \textbf{Wiki bio} &
  \textbf{Halu general} &
  \textbf{Animals} &
  \textbf{Capitals} &
  \textbf{Cities} &
  \textbf{Companies} &
  \textbf{Elements} &
  \textbf{Facts} &
  \textbf{Inventions} &
  \textbf{Overall} \\ \midrule
Dolly v2-7B  & 0.00 & 0.17 & 0.12 & 0.00 & 0.12 & 0.06 & 0.16 & 0.23 & 0.14 & 0.11 \\
Dolly v2-12B & 0.46 & 0.00 & 0.45 & 0.60 & 0.34 & 0.66 & 0.22 & 0.41 & 0.42 & 0.40 \\
Falcon-7B    & 0.34 & 0.37 & 0.52 & 0.32 & 0.39 & 0.52 & 0.41 & 0.50 & 0.38 & 0.42 \\
Falcon-40B   & 0.66 & 0.35 & 0.75 & 0.64 & 0.70 & 0.72 & 0.64 & 0.71 & 0.71 & 0.65 \\
Koala-7B     & 0.37 & 0.23 & 0.51 & 0.60 & 0.52 & 0.55 & 0.53 & 0.56 & 0.53 & 0.49 \\
Koala-13B    & 0.50 & 0.23 & 0.56 & 0.61 & 0.74 & 0.73 & 0.65 & 0.70 & 0.72 & 0.60 \\
Alpaca-7B    & 0.36 & 0.36 & 0.65 & 0.69 & 0.75 & 0.74 & 0.60 & 0.74 & 0.58 & 0.61 \\
Alpaca-13B   & 0.20 & 0.23 & 0.62 & 0.66 & 0.67 & 0.76 & 0.62 & 0.75 & 0.64 & 0.57 \\
Vicuna-7B    & 0.53 & 0.27 & 0.65 & 0.58 & 0.65 & 0.74 & 0.63 & 0.73 & 0.66 & 0.60 \\
Vicuna-13B   & 0.38 & 0.17 & 0.74 & 0.77 & 0.77 & 0.82 & 0.68 & 0.79 & 0.75 & 0.65 \\
Llama2-7B    & 0.09 & 0.22 & 0.67 & 0.58 & 0.68 & 0.76 & 0.69 & 0.77 & 0.65 & 0.57 \\
Llama2-13B   & 0.11 & 0.17 & 0.82 & 0.93 & 0.91 & 0.83 & 0.73 & 0.82 & 0.85 & 0.69 \\
Llama2-70B   & 0.51 & 0.31 & 0.78 & 0.82 & 0.75 & 0.76 & 0.65 & 0.86 & 0.70 & 0.68 \\ \midrule
GPT3.5       & 0.11 & 0.17 & 0.86 & 0.95 & 0.87 & 0.87 & 0.88 & 0.87 & 0.91 & 0.72 \\
GPT4         & 0.68 & 0.26 & 0.84 & 0.93 & 0.93 & 0.91 & 0.96 & 0.94 & 0.95 & 0.82 \\ \bottomrule
\end{tabular}
}}
\caption{The performance (F1 scores) of factuality evaluation using different LLMs over 9 datasets. The best score on each dataset is \textbf{bold} for open-source LLMs and \underline{underlined} for closed-source LLMs.}
\label{tab:llm_eval_factuality}
\end{table*}


\textbf{Result Analysis}
\section{Model-Based Metric Learning}
In this section, we describe how to train a model-based metric for both faithfulness and factuality.

\subsection{Input Prompting}
Previous model-based metrics for faithfulness~\cite{Gekhman2023TrueTeacherLF} and factuality~\cite{Azaria2023TheIS} usually model this task as a binary classification. Basically, for faithfulness, previous work~\cite{Gekhman2023TrueTeacherLF} concatenate \textit{grounding text} and \textit{generated text} into one text and feed it into a language model; for factuality, they send only \textit{generated text} to the model. Then, a standard binary text classification will be applied to finetune the model to get the model-based metric.

Different from previous models, our metric is designed for both of faithfulness and factuality evaluation. Hence, we use two kinds of prompt templates for faithfulness and factuality respectively. Specifically, for faithfulness, given the grounding text $T_{\mathrm{ground}}$ and the generated text $T_{\mathrm{gen}}$, we use the following prompt as the model input:

\textit{Input Text: }$T_{\mathrm{ground}}$ \\
\textit{Generated Text: } $T_{\mathrm{gen}}$

When faithfulness prompt template is used, our model-based metric will evaluate the consistency between $T_{\mathrm{ground}}$ and $T_{\mathrm{gen}}$. Similarly, for factuality, we have only the generated text $T_{\mathrm{gen}}$ and apply the following prompt as the model input:

\textit{Generated Text: } $T_{\mathrm{gen}}$

In this case, we expect our learned model-based metric can evaluate the consistency between the generated text and the common-sense or facts in the world.

\subsection{Model Training}
Because faithfulness is to evaluate the consistency between the grounding text and the generated text, previous methods~\cite{Gekhman2023TrueTeacherLF} fully finetune all parameters to obtain the best performance. Different from faithfulness, factuality requires that the model to evaluate the consistency between the generation and facts. Hence, we need to acquire knowledge from pre-trained parameters of language models. However, all-parameter finetuning will make the pre-trained model forget the prior knowledge from pre-training and overfit on the finetuning dataset. SAPLMA~\cite{Azaria2023TheIS} probes language models to get features for a classifier from internal hidden states. Probing is effective for factuality evaluation but it still has sub-optimal performance on faithfulness evaluation because only finetuning a multilayer perceptron limits the model capacity.

To balance the model capacity for faithfulness and prior knowledge for factuality, in this paper, we propose to employ Low-Rank Adaptation (LoRA)~\cite{Hu2021LoRALA} for model-based metric finetuning. We claim that there are two advantages of LoRA for faithfulness and factuality evaluation:
(1) LoRA finetuning has comparable performance as all-parameter finetuning~\cite{Hu2021LoRALA};
(2) LoRA finetunes only low rank metrics and keep pre-trained parameters unchanged which preserves the prior knowledge about common-sense and facts for factuality evaluation.




\section{Experiment}
\subsection{Datasets}
In this section, we will introduce the training data construction and the test data used for evaluation. 

\textbf{Training data}
In our paper, for \emph{faithfulness}, we use datasets of summarization, paraphrase and OpenQA tasks as our training data. Generally, we follow TrueTeacher~\cite{Gekhman2023TrueTeacherLF} and use synthetic data from model generation as our training corpus. Specifically, for summarization, we re-use the generated text from TrueTeacher~\cite{Gekhman2023TrueTeacherLF}; for paraphrase, we train T5~\cite{Raffel2019ExploringTL} models with different sizes (i.e.,~small, base and large) on a combined dataset including Quora Question Pairs~\cite{quora-question-pairs} and ChatGPT Paraphrases~\cite{chatgpt_paraphrases_dataset}; for OpenQA, we use Flan-T5~\cite{Chung2022ScalingIL} with different sizes (i.e.,~small, base and large) to make inference on SQUAD-v2~\cite{squad} dataset. \todo{Further details about data generation in Appendix}. After obtaining these generations, we apply the method introduced in~\Cref{sec:llm_faithful} to annotate the data with label 0 or 1. After pre-processing, the statistics of training data can be found in~\Cref{tab:train_data_stat}. 
%For fair comparison, we randomly sample 100,000 data instances in each domain respectively for model training.

For \emph{factuality}, we 


\begin{table}[h]
\centering
\small
\begin{tabular}{ccc}
\toprule
\textbf{Summarization} & \textbf{Paraphrasing} & \textbf{OpenQA}  \\ \midrule
500,000       & 296,800      & 156,762 \\ \bottomrule
\end{tabular}
\caption{Number of data instances in our synthetic training data.}
\label{tab:train_data_stat}
\end{table}


\textbf{Test data}
For test data, we use full data introduced in~\Cref{sec:benchmark} to test the faithfulness and factuality evaluation performance of different metrics.

\subsection{Baselines}
For faithfulness evaluation, we include three kinds of methods as our baselines:

(1)~\textbf{N-gram metrics} are shown to have weak correlation with faithfulness evaluation~\cite{maynez-etal-2020-faithfulness, honovich-etal-2021-q2}. Hence, we include BLEU~\cite{bleu}, Meteor~\cite{banerjee-lavie-2005-meteor} and ROUGEL~\cite{rouge}.

(2)~\textbf{General model-based metrics}
\begin{itemize}[nosep,leftmargin=*]
    \item \textbf{BERTScore}~\cite{Zhang2019BERTScoreET} aggregates token-level similarity scores between the BERT~\cite{Devlin2019BERTPO} contextual embeddings in candidate and reference text. Follow TRUE~\cite{Honovich2022TRUERF}, we use BERTScore with the DeBERTa-xl-MNLI model~\cite{He2020DeBERTaDB, nangia-etal-2017-repeval}.
    \item \textbf{BARTScore}~\cite{Yuan2021BARTScoreEG} evaluates text using probabilities of decoding from a BART model~\cite{lewis-etal-2020-bart}. We use the version finetuned on the ParaBank2 dataset~\cite{hu-etal-2019-large}.
    \item \textbf{BLEURT}~\cite{sellam-etal-2020-bleurt} is a model-based metric based on BERT for evaluating text generation. BLEURT is further pre-trained on synthetic data and then finetuned on human annotations to train a model that scores system outputs. We use the BLEURT-20 checkpoint~\cite{pu-etal-2021-learning} following TRUE.
\end{itemize}

(3)~\textbf{Model-based metrics for faithfulness}
\begin{itemize}[nosep,leftmargin=*]
    \item \textbf{FactCC}~\cite{kryscinski-etal-2020-evaluating} uses documents from CNN/DM as premise and the positive hypotheses are randomly sampled sentences from the premise. Some hypotheses are paraphrased by back-translation and injected with noise including duplicating or removing random tokens. For negative hypotheses, authors use rule-based methods, such as sentence negation and entity/pronoun/number swaps.
    \item \textbf{Falsesum}~\cite{Utama2022FalsesumGD} uses documents and summaries from CNN/DM as positive examples. To obtain negative examples, authors use the OpenIE to extract predicates and arguments in the document and corresponding summary. Predicates and arguments from the gold summary are randomly selected to be masked and infilled by predicates and arguments from document or by predictions from a infilling model.
    \item \textbf{TrueTeacher}~\cite{Gekhman2023TrueTeacherLF} generates synthetic data by training different T5 models on CNN/DM and generate summaries. To obtain supervision, they employ FLAN-PaLM 540B~\cite{Chung2022ScalingIL} to generate ``Yes'' or ``No'' to indicate if the generated text is faithful to the input doc.
\end{itemize}

For factuality evaluation, we compare our method

\subsection{Implementation Details}

\subsection{Overall Performance on Faithfulness}

\subsection{Overall Performance on Factuality}

\subsection{Fully Finetuning vs. LoRA vs. Probing}

\subsection{Multi-domain Training}

% combined vs. sum vs. paraphrase vs. qa
\section{Related Work}
\subsection{Faithfulness Evaluation}

\subsection{Factuality Evaluation}
\section{Conclusion}

\section{Limitations}
The limitations of our method are as follows:
\begin{enumerate}[nosep,leftmargin=*]
    \item 
\end{enumerate}

\section{Ethical Concerns}

% \section*{Acknowledgements}


\clearpage
% Entries for the entire Anthology, followed by custom entries
\bibliography{custom}
\bibliographystyle{acl_natbib}
\clearpage
\appendix

\section{Prompt Engineering}

\subsection{Zero-Shot vs. Few-Shot}

\subsection{Positional Bias}


\subsection{Task-Aware Prompts}

\end{document}
